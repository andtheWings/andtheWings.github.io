% Options for packages loaded elsewhere
\PassOptionsToPackage{unicode}{hyperref}
\PassOptionsToPackage{hyphens}{url}
%
\documentclass[
]{book}
\usepackage{amsmath,amssymb}
\usepackage{lmodern}
\usepackage{iftex}
\ifPDFTeX
  \usepackage[T1]{fontenc}
  \usepackage[utf8]{inputenc}
  \usepackage{textcomp} % provide euro and other symbols
\else % if luatex or xetex
  \usepackage{unicode-math}
  \defaultfontfeatures{Scale=MatchLowercase}
  \defaultfontfeatures[\rmfamily]{Ligatures=TeX,Scale=1}
\fi
% Use upquote if available, for straight quotes in verbatim environments
\IfFileExists{upquote.sty}{\usepackage{upquote}}{}
\IfFileExists{microtype.sty}{% use microtype if available
  \usepackage[]{microtype}
  \UseMicrotypeSet[protrusion]{basicmath} % disable protrusion for tt fonts
}{}
\makeatletter
\@ifundefined{KOMAClassName}{% if non-KOMA class
  \IfFileExists{parskip.sty}{%
    \usepackage{parskip}
  }{% else
    \setlength{\parindent}{0pt}
    \setlength{\parskip}{6pt plus 2pt minus 1pt}}
}{% if KOMA class
  \KOMAoptions{parskip=half}}
\makeatother
\usepackage{xcolor}
\IfFileExists{xurl.sty}{\usepackage{xurl}}{} % add URL line breaks if available
\IfFileExists{bookmark.sty}{\usepackage{bookmark}}{\usepackage{hyperref}}
\hypersetup{
  pdftitle={SIDS-Related Mortality in Cook County, IL},
  pdfauthor={Daniel P. Hall Riggins},
  hidelinks,
  pdfcreator={LaTeX via pandoc}}
\urlstyle{same} % disable monospaced font for URLs
\usepackage{color}
\usepackage{fancyvrb}
\newcommand{\VerbBar}{|}
\newcommand{\VERB}{\Verb[commandchars=\\\{\}]}
\DefineVerbatimEnvironment{Highlighting}{Verbatim}{commandchars=\\\{\}}
% Add ',fontsize=\small' for more characters per line
\usepackage{framed}
\definecolor{shadecolor}{RGB}{248,248,248}
\newenvironment{Shaded}{\begin{snugshade}}{\end{snugshade}}
\newcommand{\AlertTok}[1]{\textcolor[rgb]{0.94,0.16,0.16}{#1}}
\newcommand{\AnnotationTok}[1]{\textcolor[rgb]{0.56,0.35,0.01}{\textbf{\textit{#1}}}}
\newcommand{\AttributeTok}[1]{\textcolor[rgb]{0.77,0.63,0.00}{#1}}
\newcommand{\BaseNTok}[1]{\textcolor[rgb]{0.00,0.00,0.81}{#1}}
\newcommand{\BuiltInTok}[1]{#1}
\newcommand{\CharTok}[1]{\textcolor[rgb]{0.31,0.60,0.02}{#1}}
\newcommand{\CommentTok}[1]{\textcolor[rgb]{0.56,0.35,0.01}{\textit{#1}}}
\newcommand{\CommentVarTok}[1]{\textcolor[rgb]{0.56,0.35,0.01}{\textbf{\textit{#1}}}}
\newcommand{\ConstantTok}[1]{\textcolor[rgb]{0.00,0.00,0.00}{#1}}
\newcommand{\ControlFlowTok}[1]{\textcolor[rgb]{0.13,0.29,0.53}{\textbf{#1}}}
\newcommand{\DataTypeTok}[1]{\textcolor[rgb]{0.13,0.29,0.53}{#1}}
\newcommand{\DecValTok}[1]{\textcolor[rgb]{0.00,0.00,0.81}{#1}}
\newcommand{\DocumentationTok}[1]{\textcolor[rgb]{0.56,0.35,0.01}{\textbf{\textit{#1}}}}
\newcommand{\ErrorTok}[1]{\textcolor[rgb]{0.64,0.00,0.00}{\textbf{#1}}}
\newcommand{\ExtensionTok}[1]{#1}
\newcommand{\FloatTok}[1]{\textcolor[rgb]{0.00,0.00,0.81}{#1}}
\newcommand{\FunctionTok}[1]{\textcolor[rgb]{0.00,0.00,0.00}{#1}}
\newcommand{\ImportTok}[1]{#1}
\newcommand{\InformationTok}[1]{\textcolor[rgb]{0.56,0.35,0.01}{\textbf{\textit{#1}}}}
\newcommand{\KeywordTok}[1]{\textcolor[rgb]{0.13,0.29,0.53}{\textbf{#1}}}
\newcommand{\NormalTok}[1]{#1}
\newcommand{\OperatorTok}[1]{\textcolor[rgb]{0.81,0.36,0.00}{\textbf{#1}}}
\newcommand{\OtherTok}[1]{\textcolor[rgb]{0.56,0.35,0.01}{#1}}
\newcommand{\PreprocessorTok}[1]{\textcolor[rgb]{0.56,0.35,0.01}{\textit{#1}}}
\newcommand{\RegionMarkerTok}[1]{#1}
\newcommand{\SpecialCharTok}[1]{\textcolor[rgb]{0.00,0.00,0.00}{#1}}
\newcommand{\SpecialStringTok}[1]{\textcolor[rgb]{0.31,0.60,0.02}{#1}}
\newcommand{\StringTok}[1]{\textcolor[rgb]{0.31,0.60,0.02}{#1}}
\newcommand{\VariableTok}[1]{\textcolor[rgb]{0.00,0.00,0.00}{#1}}
\newcommand{\VerbatimStringTok}[1]{\textcolor[rgb]{0.31,0.60,0.02}{#1}}
\newcommand{\WarningTok}[1]{\textcolor[rgb]{0.56,0.35,0.01}{\textbf{\textit{#1}}}}
\usepackage{longtable,booktabs,array}
\usepackage{calc} % for calculating minipage widths
% Correct order of tables after \paragraph or \subparagraph
\usepackage{etoolbox}
\makeatletter
\patchcmd\longtable{\par}{\if@noskipsec\mbox{}\fi\par}{}{}
\makeatother
% Allow footnotes in longtable head/foot
\IfFileExists{footnotehyper.sty}{\usepackage{footnotehyper}}{\usepackage{footnote}}
\makesavenoteenv{longtable}
\usepackage{graphicx}
\makeatletter
\def\maxwidth{\ifdim\Gin@nat@width>\linewidth\linewidth\else\Gin@nat@width\fi}
\def\maxheight{\ifdim\Gin@nat@height>\textheight\textheight\else\Gin@nat@height\fi}
\makeatother
% Scale images if necessary, so that they will not overflow the page
% margins by default, and it is still possible to overwrite the defaults
% using explicit options in \includegraphics[width, height, ...]{}
\setkeys{Gin}{width=\maxwidth,height=\maxheight,keepaspectratio}
% Set default figure placement to htbp
\makeatletter
\def\fps@figure{htbp}
\makeatother
\setlength{\emergencystretch}{3em} % prevent overfull lines
\providecommand{\tightlist}{%
  \setlength{\itemsep}{0pt}\setlength{\parskip}{0pt}}
\setcounter{secnumdepth}{5}
\usepackage{booktabs}
\ifLuaTeX
  \usepackage{selnolig}  % disable illegal ligatures
\fi
\usepackage[]{natbib}
\bibliographystyle{apalike}

\title{SIDS-Related Mortality in Cook County, IL}
\author{Daniel P. Hall Riggins}
\date{2022-02-23}

\begin{document}
\maketitle

{
\setcounter{tocdepth}{1}
\tableofcontents
}
\hypertarget{sids-related-mortality-in-cook-county-il}{%
\chapter{SIDS-Related Mortality in Cook County, IL}\label{sids-related-mortality-in-cook-county-il}}

\hypertarget{abstract}{%
\section{Abstract}\label{abstract}}

Although overall rates of Sudden Infant Death Syndrome are declining, dramatic racial disparities persist in many metropolitan regions, including Cook County, IL. In this analysis, we sought to identify specific micro-regions of the county where disparities have been most severe so that public health practitioners can target future interventions with high precision. We also sought to predictively model where SIDS incidence is expected to be highest based on census-level socio-ecologic factors. We hypothesized that the microregions with highest SIDS incidence would follow patterns of historic racial segregation. Similarly, we hypothesized that factors reflecting high levels of socioeconomic disadvantage would be predictive of high SIDS incidence. However, we were interested in identifying clusters in the city where are our model would not perform well with the hopes that studying these areas of poor predictive performance might identify new factors associated with particularly harmful or protective effects.

\includegraphics[width=13.14in]{media/safe_sleep}

\emph{Image credit to \href{https://jamanetwork.com/journals/jamapediatrics/fullarticle/2599897}{JAMA Pediatrics}}

\hypertarget{import-the-data}{%
\chapter{Import the Data}\label{import-the-data}}

\hypertarget{dependencies}{%
\section{Dependencies}\label{dependencies}}

\begin{Shaded}
\begin{Highlighting}[]
\CommentTok{\# Load needed modules}
\NormalTok{box}\SpecialCharTok{::}\FunctionTok{use}\NormalTok{(}
\NormalTok{    dplyr[full\_join, glimpse, select],}
\NormalTok{    janitor[clean\_names],}
\NormalTok{    magrittr[}\StringTok{\textasciigrave{}}\AttributeTok{\%\textgreater{}\%}\StringTok{\textasciigrave{}}\NormalTok{],}
\NormalTok{    readxl[read\_xlsx],}
\NormalTok{    sf[st\_set\_geometry],}
    \AttributeTok{sids\_data\_wrangling =}\NormalTok{ .}\SpecialCharTok{/}\NormalTok{modules}\SpecialCharTok{/}\NormalTok{sids\_data\_wrangling,}
\NormalTok{    tibble[as\_tibble]}
\NormalTok{)}
\end{Highlighting}
\end{Shaded}

\hypertarget{initial-import}{%
\section{Initial import}\label{initial-import}}

First, read in the excel file that was originally shared for this project:

\begin{Shaded}
\begin{Highlighting}[]
\NormalTok{raw }\OtherTok{\textless{}{-}} 
    \CommentTok{\# Parse excel file}
    \FunctionTok{read\_xlsx}\NormalTok{(}\StringTok{"data/finaldataforanalysis3\_220121.xlsx"}\NormalTok{) }\SpecialCharTok{\%\textgreater{}\%}
    \CommentTok{\# Clean up variability in naming conventions}
    \FunctionTok{clean\_names}\NormalTok{()}
\end{Highlighting}
\end{Shaded}

\hypertarget{join-census-tract-populations}{%
\section{Join census tract populations}\label{join-census-tract-populations}}

Then, import census population data:

\begin{Shaded}
\begin{Highlighting}[]
\DocumentationTok{\#\# This is a custom function I wrote that }
\DocumentationTok{\#\# pulls data from the TidyCensus API about}
\DocumentationTok{\#\# the population count of people under five }
\DocumentationTok{\#\# years old and about spatial features }
\DocumentationTok{\#\# for each census tract. I have commented it }
\DocumentationTok{\#\# out and saved the result in an RDS file}
\DocumentationTok{\#\# so as to not make a new call to the API }
\DocumentationTok{\#\# every time this script is run. You can}
\DocumentationTok{\#\# inspect the function definition in the }
\DocumentationTok{\#\# modules folder of the source code.}

\CommentTok{\# coords\_and\_pop\_est \textless{}{-} }
\CommentTok{\#     sids\_data\_wrangling$get\_coords\_and\_pop\_est(raw)}
\CommentTok{\# }
\CommentTok{\# saveRDS(coords\_and\_pop\_est, "data/coords\_and\_pop\_est.RDS")}

\NormalTok{coords\_and\_pop\_est }\OtherTok{\textless{}{-}} \FunctionTok{readRDS}\NormalTok{(}\StringTok{"data/coords\_and\_pop\_est.RDS"}\NormalTok{)}

\CommentTok{\# Join the population counts to the imported dataframe}
\NormalTok{df }\OtherTok{\textless{}{-}} 
\NormalTok{    coords\_and\_pop\_est }\SpecialCharTok{\%\textgreater{}\%}
    \CommentTok{\# Drop geospatial features}
    \FunctionTok{st\_set\_geometry}\NormalTok{(}\ConstantTok{NULL}\NormalTok{) }\SpecialCharTok{\%\textgreater{}\%}
    \CommentTok{\# Convert to tibble format}
    \FunctionTok{as\_tibble}\NormalTok{() }\SpecialCharTok{\%\textgreater{}\%}
    \CommentTok{\# And join to raw}
    \FunctionTok{full\_join}\NormalTok{(raw)}
\CommentTok{\#\textgreater{} Joining, by = "fips"}

\CommentTok{\# Preview the data}
\FunctionTok{glimpse}\NormalTok{(df)}
\CommentTok{\#\textgreater{} Rows: 1,315}
\CommentTok{\#\textgreater{} Columns: 32}
\CommentTok{\#\textgreater{} $ fips                                 \textless{}dbl\textgreater{} 17031807500, \textasciitilde{}}
\CommentTok{\#\textgreater{} $ pop\_under\_five                       \textless{}dbl\textgreater{} 151, 192, 21,\textasciitilde{}}
\CommentTok{\#\textgreater{} $ count\_asphyxia                       \textless{}dbl\textgreater{} 0, 0, 1, 0, 0\textasciitilde{}}
\CommentTok{\#\textgreater{} $ count\_opioid\_death                   \textless{}dbl\textgreater{} 1, 7, 2, 2, 6\textasciitilde{}}
\CommentTok{\#\textgreater{} $ svi\_socioeconomic                    \textless{}dbl\textgreater{} 0.1269, 0.593\textasciitilde{}}
\CommentTok{\#\textgreater{} $ svi\_household\_composition\_disability \textless{}dbl\textgreater{} 0.1728, 0.803\textasciitilde{}}
\CommentTok{\#\textgreater{} $ svi\_minority\_language                \textless{}dbl\textgreater{} 0.7024, 0.677\textasciitilde{}}
\CommentTok{\#\textgreater{} $ svi\_housing\_transportation           \textless{}dbl\textgreater{} 0.3690, 0.528\textasciitilde{}}
\CommentTok{\#\textgreater{} $ svi\_summary\_ranking                  \textless{}dbl\textgreater{} 0.2470, 0.679\textasciitilde{}}
\CommentTok{\#\textgreater{} $ pe\_foreignborn                       \textless{}dbl\textgreater{} 31.6, 2.0, 1.\textasciitilde{}}
\CommentTok{\#\textgreater{} $ pe\_marriedmales                      \textless{}dbl\textgreater{} 62.5, 23.0, 3\textasciitilde{}}
\CommentTok{\#\textgreater{} $ pe\_marriedfemales                    \textless{}dbl\textgreater{} 56.6, 23.0, 2\textasciitilde{}}
\CommentTok{\#\textgreater{} $ pedivorcewidowedmale                 \textless{}dbl\textgreater{} 6.4, 16.9, 7.\textasciitilde{}}
\CommentTok{\#\textgreater{} $ pedivorcewidowedfemale               \textless{}dbl\textgreater{} 16.8, 34.7, 3\textasciitilde{}}
\CommentTok{\#\textgreater{} $ pelessthanhighschool                 \textless{}dbl\textgreater{} 7.1, 9.2, 8.0\textasciitilde{}}
\CommentTok{\#\textgreater{} $ highschooldiploma                    \textless{}dbl\textgreater{} 14.6, 28.4, 2\textasciitilde{}}
\CommentTok{\#\textgreater{} $ somecollege                          \textless{}dbl\textgreater{} 12.8, 26.4, 3\textasciitilde{}}
\CommentTok{\#\textgreater{} $ collegediploma                       \textless{}dbl\textgreater{} 65.5, 36.0, 3\textasciitilde{}}
\CommentTok{\#\textgreater{} $ black                                \textless{}dbl\textgreater{} 2.5, 97.4, 96\textasciitilde{}}
\CommentTok{\#\textgreater{} $ white                                \textless{}dbl\textgreater{} 58.3, 0.7, 1.\textasciitilde{}}
\CommentTok{\#\textgreater{} $ hispanic                             \textless{}dbl\textgreater{} 5.6, 0.0, 2.2\textasciitilde{}}
\CommentTok{\#\textgreater{} $ male                                 \textless{}dbl\textgreater{} 48.8, 50.8, 3\textasciitilde{}}
\CommentTok{\#\textgreater{} $ percent\_enployed                     \textless{}dbl\textgreater{} 61.6, 49.0, 4\textasciitilde{}}
\CommentTok{\#\textgreater{} $ incomelt10                           \textless{}dbl\textgreater{} 0.0, 15.7, 10\textasciitilde{}}
\CommentTok{\#\textgreater{} $ incomelt25                           \textless{}dbl\textgreater{} 3.6, 15.6, 22\textasciitilde{}}
\CommentTok{\#\textgreater{} $ incomelt50                           \textless{}dbl\textgreater{} 10.9, 15.9, 2\textasciitilde{}}
\CommentTok{\#\textgreater{} $ incomelt75                           \textless{}dbl\textgreater{} 15.7, 27.6, 1\textasciitilde{}}
\CommentTok{\#\textgreater{} $ incomegt75                           \textless{}dbl\textgreater{} 69.8, 25.3, 2\textasciitilde{}}
\CommentTok{\#\textgreater{} $ privateinsurance                     \textless{}dbl\textgreater{} 78.9, 55.5, 5\textasciitilde{}}
\CommentTok{\#\textgreater{} $ publicinsurance                      \textless{}dbl\textgreater{} 26.4, 43.5, 5\textasciitilde{}}
\CommentTok{\#\textgreater{} $ noinsurance                          \textless{}dbl\textgreater{} 2.8, 12.2, 13\textasciitilde{}}
\CommentTok{\#\textgreater{} $ spanish\_language                     \textless{}dbl\textgreater{} 6.0, 2.1, 0.7\textasciitilde{}}
\end{Highlighting}
\end{Shaded}

\hypertarget{save-for-use-in-other-chapters}{%
\section{Save for use in other chapters}\label{save-for-use-in-other-chapters}}

\begin{Shaded}
\begin{Highlighting}[]
\FunctionTok{saveRDS}\NormalTok{(df, }\AttributeTok{file =} \StringTok{"data/df.RDS"}\NormalTok{)}
\end{Highlighting}
\end{Shaded}

\hypertarget{part-exploratory-analysis}{%
\part{Exploratory Analysis}\label{part-exploratory-analysis}}

\hypertarget{mapping-sids-deaths}{%
\chapter{Mapping SIDS Deaths}\label{mapping-sids-deaths}}

\begin{verbatim}
#> PhantomJS not found. You can install it with webshot::install_phantomjs(). If it is installed, please make sure the phantomjs executable can be found via the PATH variable.
\end{verbatim}

\emph{Visit \url{http://danielriggins.com/widgets/cook_county_sids_deaths.html} for a full-screen view.}

\hypertarget{code-to-produce-the-map}{%
\section{Code to produce the map}\label{code-to-produce-the-map}}

\hypertarget{load-dependencies}{%
\subsection{Load Dependencies}\label{load-dependencies}}

\begin{Shaded}
\begin{Highlighting}[]
\NormalTok{box}\SpecialCharTok{::}\FunctionTok{use}\NormalTok{(}
\NormalTok{    dplyr[}
\NormalTok{        case\_when,}
\NormalTok{        full\_join,}
\NormalTok{        mutate,}
\NormalTok{        select}
\NormalTok{    ],}
\NormalTok{    leaflet[}
\NormalTok{        addLayersControl,}
\NormalTok{        addLegend,}
\NormalTok{        addPolygons,}
\NormalTok{        addProviderTiles, }
\NormalTok{        leaflet, }
\NormalTok{        setMaxBounds, }
\NormalTok{        setView}
\NormalTok{    ],}
\NormalTok{    leaflet.extras[addFullscreenControl],}
\NormalTok{    magrittr[}\StringTok{\textasciigrave{}}\AttributeTok{\%\textgreater{}\%}\StringTok{\textasciigrave{}}\NormalTok{],}
\NormalTok{    sf[...],}
\NormalTok{    tibble[view]}
\NormalTok{)}
\end{Highlighting}
\end{Shaded}

\hypertarget{reshape-data-for-use-in-the-map}{%
\subsection{Reshape data for use in the map}\label{reshape-data-for-use-in-the-map}}

\begin{Shaded}
\begin{Highlighting}[]
\CommentTok{\# Load SIDS death data}
\NormalTok{df }\OtherTok{\textless{}{-}}
    \CommentTok{\# Load cached geospatial features}
    \FunctionTok{readRDS}\NormalTok{(}\StringTok{"data/coords\_and\_pop\_est.RDS"}\NormalTok{) }\SpecialCharTok{\%\textgreater{}\%}
    \CommentTok{\# Join to cached dataframe}
    \FunctionTok{full\_join}\NormalTok{(}\FunctionTok{readRDS}\NormalTok{(}\StringTok{"data/df.RDS"}\NormalTok{)) }\SpecialCharTok{\%\textgreater{}\%}
    \CommentTok{\# Select ID and outcome variables}
    \FunctionTok{select}\NormalTok{(fips, count\_asphyxia) }\SpecialCharTok{\%\textgreater{}\%}
    \CommentTok{\# Turn outcome into an ordinal factor}
    \FunctionTok{mutate}\NormalTok{(}
        \AttributeTok{death\_count =} \FunctionTok{factor}\NormalTok{(}
            \FunctionTok{case\_when}\NormalTok{(}
\NormalTok{                count\_asphyxia }\SpecialCharTok{==} \DecValTok{0} \SpecialCharTok{\textasciitilde{}} \StringTok{"No Deaths"}\NormalTok{,}
\NormalTok{                count\_asphyxia }\SpecialCharTok{==} \DecValTok{1} \SpecialCharTok{\textasciitilde{}} \StringTok{"One Death"}\NormalTok{,}
\NormalTok{                count\_asphyxia }\SpecialCharTok{==} \DecValTok{2} \SpecialCharTok{\textasciitilde{}} \StringTok{"Two Deaths"}\NormalTok{,}
\NormalTok{                count\_asphyxia }\SpecialCharTok{==} \DecValTok{3} \SpecialCharTok{\textasciitilde{}} \StringTok{"Three Deaths"}\NormalTok{,}
\NormalTok{                count\_asphyxia }\SpecialCharTok{==} \DecValTok{4} \SpecialCharTok{\textasciitilde{}} \StringTok{"Four Deaths"}\NormalTok{,}
\NormalTok{                count\_asphyxia }\SpecialCharTok{==} \DecValTok{5} \SpecialCharTok{\textasciitilde{}} \StringTok{"Five Deaths"}\NormalTok{,}
\NormalTok{                count\_asphyxia }\SpecialCharTok{==} \DecValTok{6} \SpecialCharTok{\textasciitilde{}} \StringTok{"Six Deaths"}
\NormalTok{            ),}
            \AttributeTok{ordered =} \ConstantTok{TRUE}\NormalTok{,}
            \AttributeTok{levels =} \FunctionTok{c}\NormalTok{(}
                \StringTok{"No Deaths"}\NormalTok{, }
                \StringTok{"One Death"}\NormalTok{, }
                \StringTok{"Two Deaths"}\NormalTok{, }
                \StringTok{"Three Deaths"}\NormalTok{, }
                \StringTok{"Four Deaths"}\NormalTok{, }
                \StringTok{"Five Deaths"}\NormalTok{, }
                \StringTok{"Six Deaths"}
\NormalTok{            )}
\NormalTok{        )}
\NormalTok{    )}

\CommentTok{\# Configure color palette}
\NormalTok{sids\_palette }\OtherTok{\textless{}{-}} 
\NormalTok{    leaflet}\SpecialCharTok{::}\FunctionTok{colorFactor}\NormalTok{(}
        \AttributeTok{palette =} \StringTok{"magma"}\NormalTok{,}
        \AttributeTok{reverse =} \ConstantTok{TRUE}\NormalTok{,}
        \AttributeTok{levels =} \FunctionTok{c}\NormalTok{(}
                \StringTok{"No Deaths"}\NormalTok{, }
                \StringTok{"One Death"}\NormalTok{, }
                \StringTok{"Two Deaths"}\NormalTok{, }
                \StringTok{"Three Deaths"}\NormalTok{, }
                \StringTok{"Four Deaths"}\NormalTok{, }
                \StringTok{"Five Deaths"}\NormalTok{, }
                \StringTok{"Six Deaths"}
\NormalTok{            )}
\NormalTok{    )}
\end{Highlighting}
\end{Shaded}

\hypertarget{create-the-map}{%
\subsection{Create the Map}\label{create-the-map}}

\begin{Shaded}
\begin{Highlighting}[]
\CommentTok{\# Assign map to a widget object}
\NormalTok{m }\OtherTok{\textless{}{-}} \FunctionTok{leaflet}\NormalTok{(df) }\SpecialCharTok{\%\textgreater{}\%}
    \CommentTok{\# Use CartoDB\textquotesingle{}s background tiles}
    \FunctionTok{addProviderTiles}\NormalTok{(}\StringTok{"CartoDB.Positron"}\NormalTok{) }\SpecialCharTok{\%\textgreater{}\%}
    \CommentTok{\# Center and zoom the map to Cook County}
    \FunctionTok{setView}\NormalTok{(}\AttributeTok{lat =} \FloatTok{41.816544}\NormalTok{, }\AttributeTok{lng =} \SpecialCharTok{{-}}\FloatTok{87.749500}\NormalTok{, }\AttributeTok{zoom =} \DecValTok{9}\NormalTok{) }\SpecialCharTok{\%\textgreater{}\%}
    \CommentTok{\# Add button to enable fullscreen map}
    \FunctionTok{addFullscreenControl}\NormalTok{() }\SpecialCharTok{\%\textgreater{}\%}
    \CommentTok{\# Add census tract polygons colored to reflect the number of deaths}
    \FunctionTok{addPolygons}\NormalTok{(}
        \CommentTok{\# No borders to the polygons, just fill}
        \AttributeTok{stroke =} \ConstantTok{FALSE}\NormalTok{,}
        \CommentTok{\# Color according to palette above}
        \AttributeTok{color =} \SpecialCharTok{\textasciitilde{}} \FunctionTok{sids\_palette}\NormalTok{(death\_count),}
        \CommentTok{\# Group polygons by number of deaths for use in the layer control}
        \AttributeTok{group =} \SpecialCharTok{\textasciitilde{}}\NormalTok{ death\_count,}
        \CommentTok{\# Make slightly transparent}
        \AttributeTok{fillOpacity =} \FloatTok{0.7}\NormalTok{,}
        \CommentTok{\# Click on the polygon to get its ID}
        \AttributeTok{popup =} \SpecialCharTok{\textasciitilde{}} \FunctionTok{paste0}\NormalTok{(}\StringTok{"\textless{}b\textgreater{}FIPS ID:\textless{}/b\textgreater{} "}\NormalTok{, }\FunctionTok{as.character}\NormalTok{(fips))}
\NormalTok{    ) }\SpecialCharTok{\%\textgreater{}\%}
    \CommentTok{\#Add legend}
    \FunctionTok{addLegend}\NormalTok{(}
        \AttributeTok{title =} \StringTok{"Count of SIDS deaths \textless{}br\textgreater{} per census tract \textless{}br\textgreater{} from 2015{-}2019"}\NormalTok{,}
        \AttributeTok{values =} \SpecialCharTok{\textasciitilde{}}\NormalTok{ death\_count,}
        \AttributeTok{pal =}\NormalTok{ sids\_palette,}
        \AttributeTok{position =} \StringTok{"topright"}
\NormalTok{    ) }\SpecialCharTok{\%\textgreater{}\%}
    \CommentTok{\# Add ability to toggle each factor grouping on or off the map}
    \FunctionTok{addLayersControl}\NormalTok{(}\AttributeTok{overlayGroups =} \FunctionTok{c}\NormalTok{(}
                \StringTok{"No Deaths"}\NormalTok{, }
                \StringTok{"One Death"}\NormalTok{, }
                \StringTok{"Two Deaths"}\NormalTok{, }
                \StringTok{"Three Deaths"}\NormalTok{, }
                \StringTok{"Four Deaths"}\NormalTok{, }
                \StringTok{"Five Deaths"}\NormalTok{, }
                \StringTok{"Six Deaths"}
\NormalTok{            ),}
            \AttributeTok{position =} \StringTok{"topleft"}
\NormalTok{        )}
\end{Highlighting}
\end{Shaded}

\hypertarget{describing-sids-deaths}{%
\chapter{Describing SIDS Deaths}\label{describing-sids-deaths}}

Our primary outcome in the analysis is:

\texttt{count\_asphyxia} = the count of SIDS-related deaths in each census tract

This is a count-type outcome, which is typically modeled using the Poisson or Negative Binomial distributions. I will be using the Negative Binomial, since the Poisson is limited by having its variance needing to be equal to its mean.

Here we visualize the \texttt{count\_asphyxia} distribution of count frequencies.

\begin{Shaded}
\begin{Highlighting}[]
\NormalTok{DataExplorer}\SpecialCharTok{::}\FunctionTok{plot\_histogram}\NormalTok{(df}\SpecialCharTok{$}\NormalTok{count\_asphyxia)}
\end{Highlighting}
\end{Shaded}

\includegraphics{03-outcome_distribution_files/figure-latex/unnamed-chunk-2-1.pdf}

As expected, the majority of census tracts do not have any SIDS-related deaths.

It's a little hard to see the tail of the distribution so let's tabulate:

\begin{Shaded}
\begin{Highlighting}[]
\NormalTok{janitor}\SpecialCharTok{::}\FunctionTok{tabyl}\NormalTok{(df}\SpecialCharTok{$}\NormalTok{count\_asphyxia)[,}\DecValTok{1}\SpecialCharTok{:}\DecValTok{2}\NormalTok{]}
\CommentTok{\#\textgreater{}  df$count\_asphyxia    n}
\CommentTok{\#\textgreater{}                  0 1078}
\CommentTok{\#\textgreater{}                  1  176}
\CommentTok{\#\textgreater{}                  2   44}
\CommentTok{\#\textgreater{}                  3   11}
\CommentTok{\#\textgreater{}                  4    3}
\CommentTok{\#\textgreater{}                  5    2}
\CommentTok{\#\textgreater{}                  6    1}
\end{Highlighting}
\end{Shaded}


  \bibliography{book.bib,packages.bib}

\end{document}
